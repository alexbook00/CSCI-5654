\documentclass[12pt]{article}

% import a set of useful packages for math
\usepackage{amsmath, amsfonts, amssymb, mathtools}

% this package makes margins smaller
\usepackage{fullpage}

% for importing images
\usepackage{graphicx}

%%%% import any other packages here
\usepackage{algorithmic}
\usepackage{algorithm}
\usepackage{hyperref}
\hypersetup{colorlinks,urlcolor=blue}
%%%% make any other definitions here
\DeclareMathOperator*{\argmax}{argmax}
\DeclarePairedDelimiter{\ceil}{\lceil}{\rceil}


%%%%%%%%%%%%%%%%%%%%%%%%%%%%%%%%
\begin{document}

\title{Homework 3 \\
       Colorado CSCI 5654}
\author{Alex Book}
\date{February 25, 2022}
\maketitle


%%%%%%%%%%%%%%%%
\textbf{Note:} All code for this assignment (namely, problems 1, 3c, and 4) can be found \href{https://github.com/alexbook00/CSCI-5654/tree/main/Homework/HW3}{here}.

\section*{Problem 1}
\subsection*{(A)}
\subsubsection*{1.}
solution: $x_1=2, x_2=-8, x_3=4, w_1=-7, w_2=-3, w_3=2$ \\
objective row: $16+0x_4-6x_5+0x_6-3w_4-2w_5-4w_6$ \\
objective value: 16

\subsubsection*{2.}
solution: $x_1=-1, x_2=-4, x_5=0, w_3=-2, w_5=7, w_6=-3$ \\
objective row: $14+4x_3-6x_4-6x_6+2w_1+0w_2-5w_4$ \\
objective value: 14

\subsubsection*{3.}
solution: $x_1=-1, x_2=-6, x_6=2, w_4=-4, w_5=5, w_6=-1$ \\
objective row: $22-5x_3-6x_4-5x_5-3w_1+5w_2+4w_3$ \\
objective value: 22

\newpage
\subsection*{(B)}
\subsubsection*{1.}
constant column: $[$2 6 4 3 9 0$]^T$ \\
objective row: $-8-2x_1-4x_2+4x_6-2w_3+1w_4+0w_5$ \\

\subsubsection*{2.}
entering variable: $x_6$

\subsubsection*{3.}
column corresponding to entering variable: $[$-1 -2 0 1 -2 -1$]^T$

\subsubsection*{4.}
leaving variable: $w_6$

\subsubsection*{5.}
basic variable in the next dictionary: $\{x_3, x_4, x_5, w_1, w_2, x_6\}$

\newpage
\subsection*{(C)}
x = $[$0 0 0 0 1 4$]$
\begin{align*}
    w_1 &= 3 - x_1 + x_2 + x_6 &= 7 \\
    w_2 &= -1 - x_1 + x_4 + x_5 &= 0 \\
    w_3 &= -2 + x_3 + x_6 &= 2 \\
    w_4 &= 4 + x_2 + x_3 - x_4 - x_6 &= 0 \\
    w_5 &= 6 - x_1 - x_3 - x_5 - x_6 &= 1 \\
    w_6 &= -2 + x_1 + x_4 - x_5 + x_6 &= 1
\end{align*}
\\
basic variables: $x_5, x_6, w_1, w_3, w_5, w_6$ \\
non-basic variables: $x_1, x_2, x_3, x_4, w_2, w_4$ \\
\\
complete final dictionary: \\
% \begin{tabular}{r | r r r r r r r}
%     $x_5$ & $1$ & $+1x_1$ & $+0x_2$ & $+0x_3$ & $-1x_4$ & $+1w_2$ & $+0w_4$ \\
%     $x_6$ & $4$ & $+0x_1$ & $+1x_2$ & $+1x_3$ & $-1x_4$ & $+0w_2$ & $-1w_4$ \\
%     $w_1$ & $7$ & $-1x_1$ & $+2x_2$ & $+1x_3$ & $-1x_4$ & $+0w_2$ & $-1w_4$ \\
%     $w_3$ & $2$ & $+0x_1$ & $+1x_2$ & $+2x_3$ & $-1x_4$ & $+0w_2$ & $-1w_4$ \\
%     $w_5$ & $1$ & $-2x_1$ & $-1x_2$ & $-2x_3$ & $+2x_4$ & $-1w_2$ & $+1w_4$ \\
%     $w_6$ & $1$ & $+0x_1$ & $+1x_2$ & $+1x_3$ & $+1x_4$ & $-1w_2$ & $-1w_4$ \\
%     \hline
%     $z$ & $4$ & $-2x_1$ & $-2x_2$ & $+0x_3$ & $-2x_4$ & $+0w_2$ & $-1w_4$ \\
% \end{tabular}

\begin{tabular}{r | r r r r r r r}
    $x_5$ & $1$ & $+1x_1$ &  &  & $-1x_4$ & $+1w_2$ &  \\
    $x_6$ & $4$ &  & $+1x_2$ & $+1x_3$ & $-1x_4$ &  & $-1w_4$ \\
    $w_1$ & $7$ & $-1x_1$ & $+2x_2$ & $+1x_3$ & $-1x_4$ &  & $-1w_4$ \\
    $w_3$ & $2$ &  & $+1x_2$ & $+2x_3$ & $-1x_4$ &  & $-1w_4$ \\
    $w_5$ & $1$ & $-2x_1$ & $-1x_2$ & $-2x_3$ & $+2x_4$ & $-1w_2$ & $+1w_4$ \\
    $w_6$ & $1$ &  & $+1x_2$ & $+1x_3$ & $+1x_4$ & $-1w_2$ & $-1w_4$ \\
    \hline
    $z$ & $4$ & $-2x_1$ & $-2x_2$ &  & $-2x_4$ &  & $-1w_4$ \\
\end{tabular}


%%%%%%%%%%%%%%%%
\newpage
\section*{Problem 2}
\begin{align*}
    \text{max } & c^T x \\
    \text{s.t. } & Ax \leq b \\
    \\
    & \downarrow \text{ replace $x$ with the difference of two positive numbers $x^+$ and $x^-$}\\
    \\
    \text{max } & c^T(x^+ - x^-) \\
    \text{s.t. } & A(x^+ - x^-) \leq b \\
    & x^+, x^- \geq 0 \\
    \\
    & \downarrow \text{ use block matrices}\\
    \\
    \text{max } & \begin{bmatrix} c & -c \end{bmatrix} \begin{bmatrix} x^+ \\ x^- \end{bmatrix} \\
    \text{s.t. } & \begin{bmatrix} A & -A \end{bmatrix} \begin{bmatrix} x^+ \\ x^- \end{bmatrix} \leq b \\
    & \begin{bmatrix} x^+ \\ x^- \end{bmatrix} \geq \bar{0} \\
    \\
    & \downarrow \text{ primal to dual as shown in class}\\
    \\
    \text{min } & b^\text{T}y \\
    \text{s.t. } & \begin{bmatrix} A & -A \end{bmatrix}^\text{T} y \geq \begin{bmatrix} c \\ -c \end{bmatrix} \\
    & y \geq 0 \\
    \\
    & \downarrow \\
    \\
    \text{min } & b^\text{T}y \\
    \text{s.t. } & A^\text{T}y \geq c \\
    & -A^\text{T}y \geq -c \\
    & y \geq 0 \\
    \\
    \downarrow \\
    \\
    \text{min } & b^\text{T}y \\
    \text{s.t. } & A^\text{T}y = c \\
    & y \geq 0 & \square\\
\end{align*}

%%%%%%%%%%%%%%%%
\newpage
\section*{Problem 3}
\subsection*{(A)}
First we certify that the primal and dual solutions are feasible and dual-feasible, respectively. \\
\textbf{Primal:}
\begin{align*}
    x_1 - 2x_2 + x_3 = 0 \leq 0 \\
    x_1 + 2x_3 = 3 \leq 3 \\
    -x_1 + x_2 = -1.5 \leq 0 \\
    2x_2 + x_3 = 3 \leq 3 \\
    x_1 + x_3 = 3 \leq 5 \\
    x_1, x_2, x_3 = 3, 1.5, 0 \geq 0
\end{align*}
\\
\textbf{Dual:} \\
\begin{align*}
    \text{min } & 3y_2 + 3y_4 + 5y_5 \\
    \text{s.t. } & y_1 + y_2 - y_3 + y_5 \geq 3 \\
    & -2y_1 + y_3 + 2y_4 \geq 4 \\
    & y_1 + 2y_2 + y_4 + y_5 \geq -1 \\
    & y_1, y_2, y_3, y_4, y_5 \geq 0
\end{align*}
\\
\begin{align*}
    y_1 + y_2 - y_3 + y_5 = 3 \geq 3 \\
    -2y_1 + y_3 + 2y_4 = 4 \geq 4 \\
    y_1 + 2y_2 + y_4 + y_5 = 8 \geq -1 \\
    y_1, y_2, y_3, y_4, y_5 = 3, 0, 0, 5, 0 \geq 0
\end{align*}
\\
All constraints are satisfied, so the solutions are feasible. The strong duality theorem tells us that if the primal has an optimal solution $x$ then there exists a dual-feasible (and optimal) $y$ such that $c^\text{T}x = b^\text{T}y$. \\
\\
$c^\text{T}x = 3(3) + 4(1.5) - 1(0) = 15$ \\
$b^\text{T}y = 0(3) + 3(0) + 0(0) + 3(5) + 5(0) = 15$ \\
\\
Therefore, $x$ is optimal with the corresponding dual solution $y$. \square

\newpage
\subsection*{(B)}
The non-zero values in the primal solution tell us that $x_1$ and $x_2$ must both be basic variables. The non-zero values in the dual solution tell us that $y_1$ and $y_4$ must both be basic variables. We let $w$ and $p$ be the slack variables for the primal and dual, respectively. Complementary slackness tells us that $w_2$, $w_3$, $w_5$, and $p_3$ must also be basic variables. All other variables ($x_3$, $w_1$, $w_4$, $y_2$, $y_3$, $y_5$, $p_1$, and $p_2$) must be non-basic. \\
\\
\textbf{Primal} \\
Basic: \{$x_1, x_2, w_2, w_3, w_5$\}\\
Non-basic: \{$x_3, w_1, w_4$\} \\
\\
\textbf{Dual} \\
Basic: \{$y_1, y_4, p_3$\} \\
Non-basic: \{$y_2, y_3, y_5, p_1, p_2$\}

\newpage
\subsection*{(C)}
For the final dictionary to be feasible, the constant column must remain non-negative ($A_B^{-1} (b + \Delta b) \geq 0$). With the basis \{$x_1, x_2, w_2, w_3, w_5$\}, we can find: \\
\\
$A_B^{-1}$ =
$\begin{bmatrix}
    1 & 0 & 0 & 1 & 0 \\
    0 & 0 & 0 & .5 & 0 \\
    -1 & 1 & 0 & -1 & 0 \\
    1 & 0 & 1 & .5 & 0 \\
    -1 & 0 & 0 & -1 & 1
\end{bmatrix}$
\\
$A_B^{-1} (b + \Delta b) \geq 0 \xrightarrow[]{}$
$\begin{bmatrix}
    1 & 0 & 0 & 1 & 0 \\
    0 & 0 & 0 & .5 & 0 \\
    -1 & 1 & 0 & -1 & 0 \\
    1 & 0 & 1 & .5 & 0 \\
    -1 & 0 & 0 & -1 & 1
\end{bmatrix}
\begin{bmatrix}
    0 \\ 3 \\ 0 \\ 3+t \\ 5
\end{bmatrix}
\geq
\begin{bmatrix}
    0 \\ 0 \\ 0 \\ 0 \\ 0
\end{bmatrix}
\xrightarrow[]{}
\begin{bmatrix}
    3 + t \\ .5(3 + t) \\ 3 - (3 + t) \\ .5(3 + t) \\ -(3 + t) + 5
\end{bmatrix}
\geq
\begin{bmatrix}
    0 \\ 0 \\ 0 \\ 0 \\ 0
\end{bmatrix}
\xrightarrow[]{}\\
\xrightarrow[]{}
\begin{bmatrix}
    t \\ t \\ -t \\ t \\ -t
\end{bmatrix}
\geq
\begin{bmatrix}
    -3 \\ -3 \\ 0 \\ -3 \\ -2
\end{bmatrix}$
\\
\\
Therefore we have that $-3 \leq t \leq 0$. \\
\\
From lecture, we have $z^\prime = z^* + \Delta b y^*$. $z^* = 15$, $\Delta b = [$0 0 0 $t$ 0$]^\text{T}$, $y^* = [$3 0 0 5 0$]$. Therefore, $z^\prime = z^* + \Delta b y^* = 15 + 5t$.

%%%%%%%%%%%%%%%%
\newpage
\section*{Problem 4}
\begin{tabular}{l|l|l}
    Vertex & Constraints Saturated & Degenerate? \\
    \hline
    (0, 0, 0) & \{1, 2, 3, 4\} & yes \\
    (0, 2, 0) & \{2, 4, 7, 8\} & yes \\
    (1, 1, -2) & \{3, 4, 5, 8\} & yes \\
    (1, 1, 2) & \{1, 2, 6, 7\} & yes \\
    (2, 0, 0) & \{1, 3, 5, 6\} & yes \\
    (2, 2, 0) & \{5, 6, 7, 8\} & yes
\end{tabular}

\end{document}